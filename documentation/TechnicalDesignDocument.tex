\documentclass[10pt]{extarticle} % article doctype, possible font size range from 8pt to 20pt with not all being avaiable.
\setcounter{secnumdepth}{4} % Max. section depth, e.g. 3 means 4.1.1 is possible but not 4.1.1.2
\title{\huge Technical Design Document MaramaUU}
\author{Tim Hintzbergen, <nextname>    \\s1097561, <nextcode>
\\\\ICTGPb
\\Wilco Moerman}
\date{May 8th, 2018}

% include packages
\usepackage{graphicx}
\usepackage{caption}
\usepackage{mathabx}
\usepackage[margin=1.0in]{geometry} % Sets page margin, 2.0in is default.
\usepackage{titlesec}
\usepackage{hyperref}

\DeclareCaptionFormat{cancaption}{#1#2#3\par} % Normal format actually
\DeclareCaptionLabelFormat{cancaptionlabel}{#1}
\captionsetup[figure][number]{format=cancaption,labelformat=cancaptionlabel}
\graphicspath { {images/} }
\begin{document}
    \maketitle
    \thispagestyle{empty}
    \newpage
    %------------------------------------------------------------------------------------------------------------------------------------------------------- Introduction
    \newpage
    \setcounter{page}{1}
    \section {Introduction}
    introduction
    \newpage

    \tableofcontents{}
    \newpage

    %------------------------------------------------------------------------------------------------------------------------------------------------------- Requirements
    \section{Requirements}
    Functional and Non-functional requirements.
    \newpage

    %------------------------------------------------------------------------------------------------------------------------------------------------------- Architecture
    \section{Architecture}
    A high-level overview of the architecture of the system.
    A UML component diagram can give a clear overview.
    Boxes and arrows can be used during an early stage of the design.
    It should be clear where different techniques (programming languages, database systems, data formats, frameworks, ..) are used in the system.
    Class diagrams and package diagrams for the different components of your system.
    Subdivide class diagrams into smaller diagrams that focus on a particular aspect.
    Think about which details can be left out (e.g.\ private attributes and methods).
    Always make sure diagrams are readable when printed on paper.
    Only provide a textual description of classes and methods that require additional explanation.

    Sequence diagrams to show interactions between objects, if the interaction is particularly complex or involves many objects.

    Deployment diagrams to show the hardware and middleware on which the different software components run.

    Database designs, such as ERD diagrams.

    Descriptions of custom protocols, data formats etc.

    Security measures and considerations.

    Algorithm designs.

    \newpage

    %------------------------------------------------------------------------------------------------------------------------------------------------------- Design Choices
    \section{Design Choices}
    Design decisions: discuss the motivation (arguments/reasons), consequences,
    and alternatives for the decisions.
    Onderverdeeld in epics ipv user stories, makkelijker te snappen, overzichtelijker, etc. beschrijven..

    Description of custom protocols, data formats etc.\\
    Security measures and considerations.\\
    Coding conventions may be added (as appendix).\\
    Diagrams need textual explanation of context, interpreation and underlying design decisions and reasoning.\\

    \subsection{Epics}
    The view is made up of different epics.
    Epics are a collection of user stories that share a goal or functionality.

    \subsubsection{3D Camera Controls}
    \paragraph{\href{https://app.clickup.com/757520/761304/t/2e2ca}{\#2e2ca} As a MGD or player I want to zoom in/out on the world when pinching}
    Zoom functionality is already implemented and enabled by libGDX on all supported devices (Desktop, Android).

    \subsubsection{UI}
    \paragraph{\#3t5k feature X}

    \subsubsection{3D Object Controls}
    \paragraph{\#3t5k feature X}

    \newpage

    %------------------------------------------------------------------------------------------------------------------------------------------------------- Conclusion
    \section {Conclusion}

    \newpage

    %\bibliography{references}
    %\bibliographystyle{apalike} % options: apalike for apa, ieeetr for ieee
\end{document}