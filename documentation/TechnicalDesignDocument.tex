\documentclass[10pt]{extarticle} % article doctype, possible font size range from 8pt to 20pt with not all being avaiable.
\setcounter{secnumdepth}{4} % Max. section depth, e.g. 3 means 4.1.1 is possible but not 4.1.1.2
\title{\huge Technical Design Document MaramaUU}
\author{Tim Hintzbergen, <nextname>    \\s1097561, <nextcode>
\\\\ICTGPb
\\Wilco Moerman}
\date{May 8th, 2018 <TODO>}

% include packages
\usepackage{graphicx}
\usepackage{caption}
\usepackage{mathabx}
\usepackage[margin=1.0in]{geometry} % Sets page margin, 2.0in is default.
\usepackage{titlesec}
\usepackage{hyperref}

% custom commands
\newcommand{\myparagraph}[1]{\paragraph{#1}\mbox{}\\} % Without \mbox{} all newlines will be ignored, making the first sentence appear on the same line as a paragraph title.

\DeclareCaptionFormat{cancaption}{#1#2#3\par} % Normal format actually
\DeclareCaptionLabelFormat{cancaptionlabel}{#1}
\captionsetup[figure][number]{format=cancaption,labelformat=cancaptionlabel}
\graphicspath { {images/} }
\begin{document}
    \maketitle
    \thispagestyle{empty}
    \newpage
    %------------------------------------------------------------------------------------------------------------------------------------------------------- Introduction
    \newpage
    \setcounter{page}{1}
    \section {Introduction}
    This document contains all technical aspects of the product.
    It describes the requirements, the architecture, the design choices and the epics.
    The requirements talk about all functional and non-functional requirements of the product.
    The architecture explains the form of the product, offers a clear overview and talks about some minute details.
    The list of design choices illustrates and defend the various choices that have been made during the design and construction of the product.
    Finally the epics list all coherent components of the product and their submodules
    \newpage

    \tableofcontents{}
    \newpage

    %------------------------------------------------------------------------------------------------------------------------------------------------------- Requirements
    \section{Requirements}
    Functional and Non-functional requirements.
    \newpage

    %------------------------------------------------------------------------------------------------------------------------------------------------------- Architecture
    \section{Architecture}
    A high-level overview of the architecture of the system.
    A UML component diagram can give a clear overview.
    Boxes and arrows can be used during an early stage of the design.
    It should be clear where different techniques (programming languages, database systems, data formats, frameworks, ..) are used in the system.
    Class diagrams and package diagrams for the different components of your system.
    Subdivide class diagrams into smaller diagrams that focus on a particular aspect.
    Think about which details can be left out (e.g.\ private attributes and methods).
    Always make sure diagrams are readable when printed on paper.
    Only provide a textual description of classes and methods that require additional explanation.

    Sequence diagrams to show interactions between objects, if the interaction is particularly complex or involves many objects.

    Deployment diagrams to show the hardware and middleware on which the different software components run.

    Database designs, such as ERD diagrams.

    Descriptions of custom protocols, data formats etc.

    Security measures and considerations.

    Algorithm designs.

    \newpage

    %------------------------------------------------------------------------------------------------------------------------------------------------------- Design Choices
    \section{Design Choices}
    Design decisions: discuss the motivation (arguments/reasons), consequences,
    and alternatives for the decisions.
    Onderverdeeld in epics ipv user stories, makkelijker te snappen, overzichtelijker, etc. beschrijven..

    Description of custom protocols, data formats etc.\\
    Security measures and considerations.\\
    Coding conventions may be added (as appendix).\\
    Diagrams need textual explanation of context, interpreation and underlying design decisions and reasoning.\\

    It has been decided to resizes texture assets to dimensions in the power of two.
    This is to maintain backwards compatibility for OpenGL ES 1.0, to keep all functionality (some libGDX features are otherwise not supported) and it is a bit more memory efficient.\cite{libgdxpottex}

    \subsection{Epics}
    The view is made up of different epics.
    Epics are a collection of user stories that share a goal or functionality.

    \subsubsection{3D Camera Controls}
    \myparagraph{\href{https://app.clickup.com/757520/761304/t/2e2ca}{\#2e2ca} As a MGD or player I want to zoom in/out on the world when pinching}
    The camera zoom functionality is already implemented and enabled by libGDX on all supported devices (Desktop, Android).
    \myparagraph{\href{https://app.clickup.com/757520/761304/t/2e2c1}{\#2e2c1} As a MGD or player I want to rotate the camera by one-fingered swiping}
    The camera rotation functionality is already implemented and enabled by libGDX on all supported devices (Desktop, Android).

    \subsubsection{UI}
    \myparagraph{\href{https://app.clickup.com/757520/761304/t/302qb}{\#302qb} As a user I want to see the Marama logo on start-up}
    When the program is started, independent of which built is used, the splashscreen is displayed first.
    The splashscreen displays the marama-logo and fades out and finally transitions into the mainmenu.
    The fade-out time and total duration are variable and can be changed by changing their respective constants in the SplashScreen.java file.

    \subsubsection{3D Object Controls}
    \paragraph{\#3t5k feature X}

    \newpage
    %------------------------------------------------------------------------------------------------------------------------------------------------------- Build Automation
    \section{Continous Intergration}
    In this section build automation and its constituent components will be discussed.
    How we applied it, the softwere used and the targets and platforms supported are among a few.

    \newpage
    %------------------------------------------------------------------------------------------------------------------------------------------------------- References
    \bibliography{references}
    \bibliographystyle{ieeetr} % options: apalike for apa, ieeetr for ieee
\end{document}